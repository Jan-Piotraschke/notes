\section{Näherungsmethoden zum Lösen der Schrödinger-Gleichung von nicht-idealisierten Systemen}
\subsection{Störungstheorie (Störung des System):}
\begin{enumerate}
    \item Grundannahmen:
    \begin{enumerate}
        \item die Lösung des ungestörten System $\psi^\textbf{0}$ (die '0' kenntzeichnet das ungestörte System) ist bekannt
        \item die Störung ist klein
    \end{enumerate}
    \item Folgen:
    \begin{enumerate}
        \item die Störung führt allgemein zur Aufhebung der Entartung 
    \end{enumerate}
\end{enumerate} 

\subsection{Variationsprinzip}
\begin{enumerate}
    \item Grundgedanke: Grundzustandenergie $E_0$ berechnen wollen
    \begin{enumerate}
        \item eine obere Schranke für $E_0$ wird durch Hilfe einer Test-Wellenfunktion errechnet, die wir dann durch Minimierung 
        (Optimierungsprozess somit) möglichst nah in Richtung des exakten Wertes herunterrechnen.
    \end{enumerate}
\end{enumerate}

\subsection{Born-Oppenheimer Näherung (Adiabatische Näherung)}
\begin{enumerate}
    \item Anwendung: \textbf{Moleküle}
    \item Grundgedanke: Separation des Problems in schnelle $e^-$ und langsame Kerne
\end{enumerate}

\newpage