\pagestyle{headings}
\pagenumbering{arabic} 

\section{Allgemein}

\subsection{stets im Hinterkopf behalten:}
\begin{enumerate}
\item die Formeln der Quantenmechanik sind häufig nur erraten
\item \textit{„vieles spricht dafür, dass der Zufall eine fundamentale Rolle in der Natur spielen könnte und dass 
das Teilchen vor der Ortsmessung gewissermaßen selbst noch nicht weiß, wo es sich materialisieren soll.“} \\
$\rightarrow$ hier: \textbf{Wahrscheinlichkeitsinterpretation}
\item eine Wellenfunktion \underline{muss} bei der Wahrscheinlichkeitsinterpretation normiert sein (\ref{normiert})
\item physikalisches System kann nur \textbf{diskrete Werte} der Energien (z.B. Drehimpuls) annehmen
\end{enumerate}

\subsection{Postulate:}
\begin{enumerate}
\item $\psi(\vec{r}_1,\vec{r}_2, ...,\vec{r}_N,t)$ ist unser Zustand/Eigenfunktion des Teichensystem (Anzahl $N$ Teilchen), mit den Ortsvektoren $\vec{r}$
\item \textbf{gilt universell:} jede physikalisch messbare Variable besitzt einen hermitischen Operator ($\hat{A}$, $\hat{B}$, ...) (\ref{Operatoren}).\\
 if $[\hat{A},\hat{B}]=0$:
\begin{enumerate}
    \item $\hat{A}$ und $\hat{B}$ sind hintereinander messbar
    \item $\hat{A}\ket{\psi}=\lambda\ket{\psi}$ und $\hat{B}\ket{\psi}=\mu\ket{\psi}$
\end{enumerate}
\item nur die Eigenwerte ($\lambda,\mu$ etc.) sind messbar; sie sind reelle Zahlen 

\end{enumerate}



\subsection{Unterschiede Quantenmechanik (QM) zur Elektrodynamik (ED):}
\begin{enumerate}
    \item ED: \textbf{Felder} (z.B. $\vec{B}$) sind die physikalische Rolle:
    \begin{enumerate}
        \item deren Werte bestimmen an jedem Ort, welche Kraft auf die Ladung $q$ wirkt

    \end{enumerate}
    \item QM: \textbf{Potentiale} sind die physikalische Rolle
\begin{enumerate}
    \item z.B. das \textcolor{red}{Vektorpotential} $A$ von $\vec{B}$ beeinflusst die Phase einer Quantenwelle 
\end{enumerate}
\end{enumerate}

\newpage