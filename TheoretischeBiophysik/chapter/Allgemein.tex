\pagestyle{headings}
\pagenumbering{arabic} 

\section{Allgemein}

stets im Hinterkopf behalten:
\begin{enumerate}
\item die Formeln der Quantenmechanik sind häufig nur erraten
\item „vieles spricht dafür, dass der Zufall eine fundamentale Rolle in der Natur spielen könnte und dass 
das Teilchen vor der Ortsmessung gewissermaßen selbst noch nicht weiß, wo es sich materialisieren soll.“ $\rightarrow$ hier: \textbf{Wahrscheinlichkeitsinterpretation}
\end{enumerate}

Postulate:
\begin{enumerate}
\item messbare Variablen haben einen hermitischen Operator ($\hat{A}$, $\hat{B}$, ...) 
\item if $[\hat{A},\hat{B}]=0$:
\begin{enumerate}
    \item $\hat{A}$ und $\hat{B}$ sind hintereinander messbar
    \item $\hat{A}\ket{\psi}=\lambda\ket{\psi}$ und $\hat{B}\ket{\psi}=\mu\ket{\psi}$
\end{enumerate}
\item nur die Eigenwerte ($\lambda,\mu$ etc.) sind messbar; sie sind reelle Zahlen 

\end{enumerate}

physikalisches System kann nur \textbf{diskrete Werte} der Energien (z.B. Drehimpuls) annehmen:
\begin{enumerate}
    \item Leiteroperatoren $\hat{a}_\pm$ verringern/erhöhen die Eigenwerte in diskrete Schritte 
\end{enumerate}

Unterschiede Quantenmechanik (QM) zur Elektrodynamik (ED):
\begin{enumerate}
    \item ED: \textbf{Felder} (z.B. $\vec{B}$) sind die physikalische Rolle:
    \begin{enumerate}
        \item deren Werte bestimmen an jedem Ort, welche Kraft auf die Ladung $q$ wirkt

    \end{enumerate}
    \item QM: \textbf{Potentiale} sind die physikalische Rolle
\begin{enumerate}
    \item z.B. das \textcolor{red}{Vektorpotential} $A$ von $\vec{B}$ beeinflusst die Phase einer Quantenwelle 
\end{enumerate}
\end{enumerate}