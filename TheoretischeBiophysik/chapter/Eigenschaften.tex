\section{Arbeiten mit $\psi$}

\subsection{Grundregeln für Verwendung von $\psi$:}
\begin{enumerate}
    \item die Energieniveaus spalten sich im homogenen $\vec{B}$ auf: Zeeman-Effekt genannt
    \item klassisch erlaubt: $V<E$ $\rightarrow$ "oszillierendes Verhalten von $\psi$"
    \item klassisch verboten: $V>E$ $\rightarrow$ Tunneleffekt kann auftreten (u.a. bei einer \textbf{Potentialbarriere} (\ref{Potentialbarriere}))
    \item $\psi$ besitzt Quantenzahlen: $n$ bestimmt $l$, $l$ bestimmt $m$ $\rightarrow$ allgemein somit: $\psi_{nlm}$
    \begin{enumerate}
        \item Quantenzahl $n$ (legt die diskreten \textit{Energiestufen} fest): $n=1,2,3,...$
        \item Bahnquantenzahl $l$ (gehört zu $L^2$; bestimmt die \textit{Form} von $\psi$ (\ref{FormVonPsi})):\\
        $0\leq l\leq n-1$ (\textbf{Anmerkung}: können nur ganzzahlig oder halbzahlig sein)
        \item magnetische Quantenzahl $m$ (gehört zu $L_z$; bestimmt die \textit{Orientierung} von $\psi$ und magn. Eigenschaften):\\
        $-l\leq m \leq +l$ (\textbf{Anmerkung}: in ganzzahligen Schritten)
    \end{enumerate}
\end{enumerate}

\subsection{Photonen:}
\begin{enumerate}
    \item Wellenpackete $=$ Überlagerung von ebenen Wellen
\end{enumerate}

\subsection{Messung von $\psi$ mittels hermitische Operatoren:} \label{Operatoren}
\begin{enumerate}
    \item Ort: $\hat{x}$
    \item Impuls: $\hat{p}$
    \item Gesamtenergie: $\hat{H}$
    \item Drehimpuls: $\hat{L}_i=\sum_{jik}\epsilon_{ijk}\hat{x}_j\hat{p}_k$ (\ref{Drehimpuls})
    \begin{enumerate}
        \item "Betragsinformation": $\hat{L}^2$
        \item (teilweise) "Richtungsinformation": $\hat{L}_z$
    \end{enumerate}

\end{enumerate}
\newpage

\subsection{die 2 Drehimpulse $L$; in 3D auftretend (\textbf{Einheit}: [$\hbar$]):} \label{Drehimpuls}
\begin{enumerate}
    \item nice to know:
    \begin{enumerate}
        \item die Eigenwerte von $\hat{L}^2$ sind $\lambda=\hbar^2 l (l+1)$
        \item die Eigenwerte von $\hat{L}_z$ sind $\lambda=\hbar m$, mit Quantenzahl $m$
        \item $[\hat{L}^2,\hat{L}_z]=0$
        \item $\hat{L}_\pm=\hat{L}_x \pm i\hat{L}_y$
        \begin{enumerate}
            \item $\hat{L}_\pm$ verändert den Eigenwert von $\hat{L}_z$ um $\pm \hbar$, was die Einheit des Drehimpuls entspricht!
            \item $\hat{L}_\pm$ verändert \underline{nicht} den Eigenwert von $\hat{L}^2$
        \end{enumerate}
    \end{enumerate}
    \item Bahndrehimpuls:
    \begin{enumerate}
        \item nimmt nur ganzzahlige Werte für $l$ ein: $l=0,1,2,3,...$
        \begin{enumerate}
            \item $l=0\rightarrow$ s-Orbitale
            \item $l=1\rightarrow$ p-Orbitale
            \item $l=2\rightarrow$ d-Orbitale
        \end{enumerate}
        \item zu jedem $l$ gibt es $(2l+1)$ mögliche Zustände für $\psi$, die entartet sind \label{FormVonPsi}
    \end{enumerate}
    \item Spin $s$: eine Art ewige Eigendrehung der Teilchen \\
    \textbf{Hinweis:} algebraische Theorie identisch zum Bahndrehimpuls d.h. Auf-/Absteigeoperatoren, Kommutator etc.
\begin{enumerate}
    \item aus dem Spin-Statistik-Theorem:
\begin{enumerate}
    \item Fermionen (z.B. $e^-$, Proton, Neutron): s=0.5  $\rightarrow$ $\psi$ ist antisymmetrisch
    \begin{enumerate}
        \item Fermionen sind „Einzelgänger“ d.h. sie können nicht am selben Ort sein
    \end{enumerate}
    \item Bosonen (z.B. Photonen): s=1 $\rightarrow$ $\psi$ ist symmetrisch 
    \begin{enumerate}
        \item Bosonen sind „Herdentiere“ (lieben Gleichschritt) d.h. z.B.  können sich unzählige
        Photonen zusammentun und gemeinsam eine schwingende elektromagnetische Welle ausbilden 
    \end{enumerate}
\end{enumerate}
\item $s=0.5$:
\begin{enumerate}
    \item $s=0.5$ $\rightarrow$ nur 2 mögliche Eigenzustände $\ket{m_s}$: (siehe Stern-Gerlach Versuch):\\
    $\ket{m_s=-0.5}\equiv\ket{\downarrow}=\begin{pmatrix} 0\\ 1 \end{pmatrix}$  (spin-down) \\
    und $\ket{m_s=+0.5}\equiv\ket{\uparrow}=\begin{pmatrix} 1\\ 0\end{pmatrix}$ (spin-up)
\end{enumerate}
\end{enumerate}
\end{enumerate}

\subsection{Wechselwirkung Strahlung - Materie (Moleküle)}
\begin{enumerate}
    \item die einfallende elektromagnetische Welle stellt eine Störung des Systems $\psi$ dar
    \item Übergänge zwischen den Bohrschen Energieniveaus sind bei $\Delta m=0,\pm 1$ bzw. $\Delta l=\pm 1$ möglich
\end{enumerate}

\newpage