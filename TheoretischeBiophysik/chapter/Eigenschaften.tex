\section{$\psi$ in der Praxis}
Photonen:
\begin{enumerate}
    \item Wellenpackete $\rightarrow$ Überlagerung von ebenen Wellen
\end{enumerate}

die 2 Drehimpulse $L$ (\textbf{Einheit}: [$\hbar$]):
\begin{enumerate}
    \item Bahndrehimpuls
    \item Spin $s$: eine Art ewige Eigendrehung der Teilchen
\begin{enumerate}
    \item aus dem Spin-Statistik-Theorem:
\begin{enumerate}
    \item Fermionen (z.B. $e^-$, Proton, Neutron): s=0.5  $\rightarrow$ $\psi$ ist antisymmetrisch
    \begin{enumerate}
        \item Fermionen sind „Einzelgänger“ d.h. sie können nicht am selben Ort sein
    \end{enumerate}
    \item Bosonen (z.B. Photonen): s=1 $\rightarrow$ $\psi$ ist symmetrisch 
    \begin{enumerate}
        \item Bosonen sind „Herdentiere“ (lieben Gleichschritt) d.h. z.B.  können sich unzählige
        Photonen zusammentun und gemeinsam eine schwingende elektromagnetische Welle ausbilden 
    \end{enumerate}
\end{enumerate}
\item $s=0.5$:
\begin{enumerate}
    \item $s=0.5$ $\rightarrow$ $m_s=-0.5$ und $m_s=+0.5$
\end{enumerate}
\end{enumerate}
\end{enumerate}
