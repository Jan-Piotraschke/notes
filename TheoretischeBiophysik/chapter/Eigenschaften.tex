\section{$\psi$ in der Praxis}

\subsection{Grundregeln für Verwendung von $\psi$:}
\begin{enumerate}
    \item die Energieniveaus spalten sich im homogenen $\vec{B}$ auf: Zeeman-Effekt genannt
    \item klassisch erlaubt: $V<E$ $\rightarrow$ "oszillierendes Verhalten von $\psi$"
    \item klassisch verboten: $V\geq E$ $\rightarrow$ Tunneleffekt kann auftreten 
    \item $\psi$ besitzt Quantenzahlen: $n$ bestimmt $l$, $l$ bestimmt $m$
    \begin{enumerate}
        \item $n$ (legt die diskreten Energiestufen fest): $n=1,2,3,...$
        \item $l$ (gehört zu $L^2$): $0\leq l\leq n-1$
        \item $m$ (gehört zu $L_z$): $m=-l_{max},...,+l_{max}$ 
    \end{enumerate}
\end{enumerate}

\subsection{Photonen:}
\begin{enumerate}
    \item Wellenpackete $=$ Überlagerung von ebenen Wellen
\end{enumerate}

\subsection{Messung von $\psi$ mittels hermitische Operatoren:} \label{Operatoren}
\begin{enumerate}
    \item Ort: $\hat{x}$
    \item Impuls: $\hat{p}$
    \item Gesamtenergie: $\hat{H}$
    \item Drehimpuls: $\hat{L}_i=\sum_{jik}\epsilon_{ijk}\hat{x}_j\hat{p}_k$ (\ref{Drehimpuls})
    \begin{enumerate}
        \item "Betragsinformation": $\hat{L}^2$
        \item (teilweise) "Richtungsinformation": $\hat{L}_z$
    \end{enumerate}

\end{enumerate}

\subsection{die 2 Drehimpulse $L$ (\textbf{Einheit}: [$\hbar$]):} \label{Drehimpuls}
\begin{enumerate}
    \item Bahndrehimpuls
    \item Spin $s$: eine Art ewige Eigendrehung der Teilchen
\begin{enumerate}
    \item aus dem Spin-Statistik-Theorem:
\begin{enumerate}
    \item Fermionen (z.B. $e^-$, Proton, Neutron): s=0.5  $\rightarrow$ $\psi$ ist antisymmetrisch
    \begin{enumerate}
        \item Fermionen sind „Einzelgänger“ d.h. sie können nicht am selben Ort sein
    \end{enumerate}
    \item Bosonen (z.B. Photonen): s=1 $\rightarrow$ $\psi$ ist symmetrisch 
    \begin{enumerate}
        \item Bosonen sind „Herdentiere“ (lieben Gleichschritt) d.h. z.B.  können sich unzählige
        Photonen zusammentun und gemeinsam eine schwingende elektromagnetische Welle ausbilden 
    \end{enumerate}
\end{enumerate}
\item $s=0.5$:
\begin{enumerate}
    \item $s=0.5$ $\rightarrow$ $m_s=-0.5$ und $m_s=+0.5$
\end{enumerate}
\end{enumerate}
\end{enumerate}



\subsection{Störung des System (Störungstheorie):}
\begin{enumerate}
    \item Grundannahmen:
    \begin{enumerate}
        \item die Lösung des ungestörten System ist bekannt
        \item die Störung ist klein
    \end{enumerate}
\end{enumerate}