\section{statistische Quantenwelle $\psi$}

\subsection{zum Verständnis:}
\begin{enumerate}
    \item $\psi$ ist unser Zustand/Eigenfunktion des Teichens
    \item die Wahrscheinlickeitsaussagen sind die Lösung der Schrödinger-Gleichung (\ref{schroedinger})

    \item ein $e^-$ hat in einem Atom keine Flugbahn. $\psi$ der $e^-$ sagt nur wie wahrscheinlich es ist,
    irgendwo ein $e^-$  anzutreffen, wenn man es experimentell misst
    \item durch Messung \textit{kollabiert} $\psi$ zu einem Peak am gemessenen Wert 
    \item \textbf{?} es gibt diskrete Energien aber kontinuierliche Orte 
\end{enumerate}

\subsection{Bedingungen aus der Statistik:}
\begin{enumerate}
    \item Mittelwert $\braket{f(x)}=\int_{-\infty}^{\infty} f(x)\rho(x) \,dx$
    \item da $\int_{-\infty}^{\infty} \rho(x) \,dx=1$ muss $\psi$ normiert sein, damit 
    das Teilchen nicht zu einer Wahrscheinlichkeit ungleich 100 $\%$ im ganzen Raum vorfindbar ist
\end{enumerate}

\subsection{Schrödinger-Gleichung:}\label{schroedinger}
\begin{enumerate}
    \item die Schrödinger-Gleichung hängt nur von zwei Funktionen ab: $\psi$ und Potential $V$
    \item (allg.) zeitabhängige: $i\hbar\frac{\partial\psi\left(x,t\right)}{\partial t}=-\frac{\hbar^{2}}{2m}\frac{\partial^{2}\psi\left(x,t\right)}{\partial x^{2}}+V\cdot\psi\left(x,t\right)$
    \begin{enumerate}
        \item Separationsansatz: $\psi(x,t)=\varphi(x)\cdot f(t)$
    \end{enumerate}

    \item zeitunabhängige: $-\frac{\hbar^{2}}{2m}\frac{\partial^{2}\psi\left(x\right)}{\partial x^{2}}+V(x)\cdot\psi\left(x\right)=E\psi\left(x\right)$ 
    \begin{enumerate}
        \item auch geschrieben als $\hat{H}\psi=E\psi$
    \end{enumerate}

    \item es gibt Modellsysteme für die Form des Potential:
    \begin{enumerate}
        \item harmonischer Oszillator:
        \begin{enumerate}
            \item Leiteroperatoren $\hat{a}_\pm$ sind Lösungsansätze der Schrödinger-Gleichung und 
            verringern/erhöhen die Energien in diskrete Schritte:
            \begin{enumerate}
                \item die Lösungen davon sind die Hermite-Funktionen
            \end{enumerate}
        \end{enumerate}
    \end{enumerate}
\end{enumerate}


\newpage