\section{statistische Quantenwelle $\psi$}

zum Verständnis:
\begin{enumerate}
    \item $\psi$ ist unser Zustand/Eigenfunktion
    \item die Wahrscheinlickeitsaussagen sind die Lösung der Schrödinger-Gleichung
    \item ein $e^-$ hat in einem Atom keine Flugbahn. $\psi$ der $e^-$ sagt nur wie wahrscheinlich es ist,
    irgendwo ein $e^-$  anzutreffen, wenn man es experimentell misst
    \item durch Messung \textit{kollabiert} $\psi$ zu einem Peak am gemessenen Wert 
    \item \textbf{?} es gibt diskrete Energien aber kontinuierliche Orte 
\end{enumerate}

Bedingungen aus der Statistik:
\begin{enumerate}
    \item Ein Teilchen wird repräsentiert durch eine Wellenfunktion $\psi(x,t)$
    \item Mittelwert $\braket{f(x)}=\int_{-\infty}^{\infty} f(x)\rho(x) \,dx$
    \item da $\int_{-\infty}^{\infty} \rho(x) \,dx=1$ muss $\psi$ normiert sein, damit 
    das Teilchen nicht zu einer Wahrscheinlichkeit ungleich 100 $\%$ im ganzen Raum vorfindbar ist
\end{enumerate}

zeitunabhängige Schrödinger-Gleichung: $-\frac{\hbar^{2}}{2m}\frac{\partial^{2}\psi\left(x\right)}{\partial x^{2}}+V(x)\cdot\psi\left(x\right)=E\psi\left(x\right)$ 
\begin{enumerate}
    \item auch geschrieben als $\hat{H}\psi=E\psi$
\end{enumerate}

(allg.) zeitabhängige Schrödinger-Gleichung: $i\hbar\frac{\partial\psi\left(x,t\right)}{\partial t}=-\frac{\hbar^{2}}{2m}\frac{\partial^{2}\psi\left(x,t\right)}{\partial x^{2}}+V\cdot\psi\left(x,t\right)$
\begin{enumerate}
    \item Separationsansatz: $\psi(x,t)=\varphi(x)\cdot f(t)$

\end{enumerate}

\newpage