\section{Quantenwelle $\psi$ (unser Zustand)}

zum Verständnis:
\begin{enumerate}
    \item ein $e^-$ hat in einemAtom keine Flugbahn. $\psi$ der $e^-$ sagt nur wie wahrscheinlich es ist,
    irgendwo ein $e^-$  anzutreffen, wenn man es experimentell misst
\end{enumerate}

Drehimpulse:
\begin{enumerate}
    \item Bahndrehimpuls
    \item Spin $s$: eine Art ewige Eigendrehung der Teilchen
\begin{enumerate}
    \item Spin-Statistik-Theorem:
\begin{enumerate}
    \item Fermionen (z.B. $e^-$): s=0.5  $\rightarrow$ $\psi$ ist antisymmetrisch
    \begin{enumerate}
        \item Fermionen sind „Einzelgänger“ d.h. sie können nicht am selben Ort sein
    \end{enumerate}
    \item Bosonen (z.B. Photonen): s=1 $\rightarrow$ $\psi$ ist symmetrisch 
    \begin{enumerate}
        \item Bosonen sind „Herdentiere“ (lieben Gleichschritt) d.h. z.B.  können sich unzählige
        Photonen zusammentun und gemeinsam eine schwingende elektromagnetische Welle ausbilden 
    \end{enumerate}
\end{enumerate}
\end{enumerate}
\end{enumerate}
