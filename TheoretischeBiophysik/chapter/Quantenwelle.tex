\section{statistische Quantenwelle $\psi$}

\subsection{zum Verständnis:}
\begin{enumerate}
    \item die Wahrscheinlickeitsaussagen sind die Lösung der Schrödinger-Gleichung (\ref{schroedinger})
    \item ein $e^-$ hat in einem Atom keine Flugbahn. $\psi$ der $e^-$ sagt nur wie wahrscheinlich es ist,
    irgendwo ein $e^-$  anzutreffen, wenn man es experimentell misst
    \item durch Messung \textit{kollabiert} $\psi$ zu einem Peak am gemessenen Wert 
    \item es gilt die Heisenberg'sche Unschärferelation $\sigma_x\sigma_p\geq\frac{\hbar}{2}$ (siehe \ref{Varianz})
    \item \textbf{?} es gibt diskrete Energien aber kontinuierliche Orte 
\end{enumerate}

\subsection{Bedingungen aus der Statistik:}
\begin{enumerate}
    \item allgemein: $\braket{\psi_m|\psi_n}=\int \psi_m^*\psi_n \,dr=\delta_{mn}$ mit Wahrscheinlichkeitsdichte $|\psi(r,t)|^2=\rho(r,t)$
    \item da $\int_{-\infty}^{\infty} \rho(r) \,dr=1$ muss $\psi$ normiert sein, damit 
    das Teilchen nicht zu einer Wahrscheinlichkeit ungleich 100 $\%$ im ganzen Raum vorfindbar ist
    \item Erwartungswert/Mittelwert $\bar{A}\equiv\braket{A}=\braket{\psi|\hat{A}\psi}$
    \item Varianz $\sigma^2=\braket{A^2}-\braket{A}^2$ \label{Varianz}
\end{enumerate}

\subsection{Schrödinger-Gleichung:}\label{schroedinger}
\begin{enumerate}
    \item die Schrödinger-Gleichung hängt nur von zwei Funktionen ab: $\psi$ und Potential $V$
    \item (allg.) zeitabhängige: $i\hbar\frac{\partial\psi\left(x,t\right)}{\partial t}=-\frac{\hbar^{2}}{2m}\frac{\partial^{2}\psi\left(x,t\right)}{\partial x^{2}}+V\cdot\psi\left(x,t\right)$
    \begin{enumerate}
        \item Separationsansatz: $\psi(x,t)=\varphi(x)\cdot f(t)$
    \end{enumerate}


    \item zeitunabhängige: $-\frac{\hbar^{2}}{2m}\frac{\partial^{2}\psi\left(x\right)}{\partial x^{2}}+V(x)\cdot\psi\left(x\right)=E\psi\left(x\right)$ 
    \begin{enumerate}
        \item auch geschrieben als $\hat{H}\psi=E\psi$
        \item es gibt Modellsysteme für die Form des Potentials $V$:
        \begin{enumerate}
            \item freies Teichen (d.h. $V=0$):
            \begin{enumerate}
                \item $-\frac{\hbar^{2}}{2m}\frac{\partial^{2}\psi\left(x\right)}{\partial x^{2}}=E\psi\left(x\right)$
                \item \textbf{?} (NOTE: gehört das nicht zur zeitabhängigen Schrödinger-Gleichung?) Teilchen ist eine ebene Welle ($\psi$ kann zum Zeitpunkt $t=0$ "lokalisiert" werden
                und dort normiert werden)
                \item die Geschwindigkeit ist durch die Gruppengeschwindigkeit $v_g=\frac{\hbar k}{m}=\frac{d}{dk}\omega$ und \underline{nicht}
                durch die Phasengeschwindigkeit von $\psi$ gegeben
                \item \textbf{Model vom unendlichen Potentialtopf anwendbar}  
            \end{enumerate}
            \item harmonischer Oszillator ($V=\frac{1}{2}m\omega^2x^2$):
            \begin{enumerate}
                \item $E_n=\hbar\omega(n+\frac{1}{2})$ mit $n=0,1,2,...$
                \item $E\psi=\hbar\omega(\hat{a}_\pm\hat{a}_\mp\pm\frac{1}{2})\psi$
                \item Leiteroperatoren $\hat{a}_\pm$ sind Lösungsansätze der Schrödinger-Gleichung und 
                verringern/erhöhen die Energien in diskrete Schritte: \\ die Lösungen davon sind die Hermite-Funktionen
            \end{enumerate}
        \end{enumerate}
    \end{enumerate}

\end{enumerate}



\newpage